\twocolumn[{%
    \begin{frontmatter}
    
    %% Title, authors and addresses
    \title{\vspace{-0.5cm}Computational and experimental insights into the \\
    chemosensory navigation of \textit{Aedes aegypti} mosquito larvae}
    \author[1]{Eleanor K. Lutz}
    \author[2]{Tjinder S. Grewal}
    \author[1]{Jeffrey A. Riffell}\corref{cor1}
    \address[1]{Department of Biology, University of Washington, Box 351800, Seattle, WA 98195, USA}
    \address[2]{Department of Biochemistry, University of Washington, Box 357350, Seattle, WA 98195, USA}
    
    %% Text of abstract
    \begin{abstract}
Mosquitoes are prolific disease vectors that affect public health around the world. Although many studies have investigated search strategies used by host-seeking adult mosquitoes, little is known about larval search behavior. Larval behavior affects adult body size and fecundity, and thus the capacity of individual mosquitoes to find hosts and transmit disease. Understanding vector survival at all life stages is crucial for improving disease control. In this study we use experimental and computational methods to investigate the chemical ecology and search behavior of \textit{Aedes aegypti} mosquito larvae. We show that larvae do not respond to several olfactory cues used by adult \textit{Ae. aegypti} to assess larval habitat quality, but perceive microbial RNA as a potent foraging attractant. Second, we demonstrate that \textit{Ae. aegypti} larvae use a strategy consistent with chemokinesis, rather than chemotaxis, to navigate chemical gradients. Using computational modeling, we further show that chemokinesis is more efficient than chemotaxis for avoiding repellents in ecologically relevant larval habitat sizes. Finally, we use experimental observations and computational analyses to demonstrate that larvae respond to starvation pressure by optimizing exploration behavior. Our results identify key characteristics of foraging behavior in a disease vector mosquito, including the identification of a surprising foraging attractant and an unusual behavioral mechanism for chemosensory preference. In addition to implications for better understanding and control of disease vectors, this work establishes mosquito larvae as a tractable model for chemosensory behavior and navigation.
    
    \vspace{0.2cm}
    \footnotesize\textit{Keywords:} Mosquito, Behavior, \textit{\textit{Aedes aegypti}}, Larvae, Chemotaxis, Chemosensation
    \vspace{0.05cm}
    
    \end{abstract}
    \end{frontmatter}
}]